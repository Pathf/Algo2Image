\documentclass{article}
\usepackage[utf8]{inputenc} %cp1252 pour Windows, utf8 pour Linux
\usepackage[T1]{fontenc}
\usepackage{lmodern}
\usepackage{graphicx}
\usepackage[frenchb]{babel}
\usepackage{hyperref}
\usepackage[table,xcdraw]{xcolor}
\usepackage{float}

\newcommand{\info}{\texttt}
\newcommand{\qt}{\info{quatree}}

\title{Algorithmique et Structure de Données 2\\
Rapport Projet 2}
\author{Valentin \bsc{Hénique} \and Corentin \bsc{Chédotal}}
\date{02 Mai 2016}

\begin{document}

\maketitle

\section{Introduction}

Dans le cadre de l'Unité d'Enseignement X4I0030 intitulée "Algorithmique et Structure de Données 2" nous avons été amené à produire un second et dernier projet. Celui-ci consiste en la réalisation d'un algorithme de compression d'images bitmap. Pour ce faire nous avions à notre disposition un arbre de recherche d'arité 4 : le \qt. Il était donc demandé de représenter les images à travers ces arbres particuliers en suivant une méthode spécifique expliquée dans le sujet. Ceci fait il nous fallait appliquer nos algorithmes de compressions et comparer divers réglages de ceux-ci.
Le langage de programmation demandé étant le C++.\\
Ce rapport expliquera donc comment ces algorithmes ont été mis en place ainsi que les contraintes mémorielles et temporaires de ceux-ci. Nous comparerons d'ailleurs différents fonctionnements de ceux-ci dans des environnements de test spécifiques qui eux aussi seront expliqués.

\section{Implémentation}

PLACEHOLDER

    \subsection{Méthodes employées}

    PLACEHOLDER
    
    \subsection{Encombrement mémoire}
    
    PLACEHOLDER
    
    \subsection{Complexités temporelles}

    PLACEHOLDER
    
        \subsubsection{Complexités temporelles théoriques}
        
        PLACEHOLDER
        
        \subsubsection{Complexités temporelles expérimentales}
        
        PLACEHOLDER
        
            \paragraph{Protocole expérimental}
            
            PLACEHOLDER
        
            \paragraph{Base de jeux de test et pertinence de ceux-ci}
            
            PLACEHOLDER
            
            \paragraph{Résultats}
            
            PLACEHOLDER

\section{Conclusion}

PLACEHOLDER

\newpage
\appendix

\section*{Annexe}

PLACEHOLDER ANNEXE

\newpage
\tableofcontents

\end{document}
